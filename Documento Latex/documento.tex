\documentclass[12pt]{article}
\usepackage[utf8]{inputenc}
\usepackage[spanish]{babel}
\usepackage{amsmath, amssymb}
\usepackage{graphicx}
\usepackage{geometry}
\usepackage{fancyhdr}
\usepackage{enumitem}
\usepackage{hyperref}
\geometry{margin=1in}
\pagestyle{fancy}
\fancyhf{}
\rhead{Gobernanza de Modelos}
\lhead{Grupo Financiero}
\cfoot{\thepage}

\title{\textbf{Gobernanza de Modelos del Área de Riesgos}}
\author{Grupo Financiero - Subgerencia de Ciencia de Datos}
\date{\today}

\begin{document}

\maketitle

\tableofcontents

\newpage

\section{Introducción y Propósito}
La presente política establece el marco de \textbf{Gobernanza de Modelos} para el Área de Riesgos del Grupo Financiero. Su objetivo es definir principios, procesos y responsabilidades que aseguren que los modelos sean:
\begin{itemize}
  \item Conceptualmente sólidos,
  \item Validados rigurosamente,
  \item Monitoreados durante su ciclo de vida,
  \item Usados apropiadamente para la toma de decisiones,
  \item Transparentes y auditables.
\end{itemize}

\section{Alcance}
Esta política aplica a todos los modelos utilizados en la gestión de riesgos, incluyendo pero no limitados a:
\begin{itemize}
  \item Modelos estadísticos tradicionales (regresión, scoring, segmentación)
  \item Modelos regulatorios (PD, LGD, EAD, IFRS 9)
  \item Modelos de Machine Learning (Random Forest, XGBoost, etc.)
  \item Modelos de Deep Learning (Redes Neuronales, LSTM, etc.)
  \item Modelos híbridos y de simulación económica
\end{itemize}

\section{Roles y Responsabilidades}
\begin{itemize}
  \item \textbf{Primera Línea (Desarrolladores / Usuarios)}: Desarrollo, documentación, pruebas y uso.
  \item \textbf{Segunda Línea (Gobernanza de Modelos)}: Validación independiente, inventario, monitoreo, estándares.
  \item \textbf{Tercera Línea (Auditoría Interna)}: Verificación del cumplimiento del marco de gobernanza.
  \item \textbf{Comité de Modelos}: Instancia de aprobación, revisión de excepciones y seguimiento de riesgos.
\end{itemize}

\section{Ciclo de Vida del Modelo}
\subsection{1. Identificación e Inventario}
Todo modelo debe ser registrado con información mínima: propósito, responsables, tipo de modelo, criticidad, frecuencia de uso y variables clave.

\subsection{2. Desarrollo}
Debe seguir buenas prácticas: selección de variables, definición de datasets, replicabilidad, separación de datos (entrenamiento/prueba), control de sesgos y justificación del algoritmo.

\subsection{3. Validación Independiente}
Revisión técnica rigurosa por una unidad distinta al desarrollador. Incluye:
\begin{itemize}
  \item Pruebas de robustez y sensibilidad,
  \item Validación estadística (AUC, MAPE, RMSE, etc.),
  \item Evaluación de overfitting,
  \item Transparencia y explicabilidad (SHAP, LIME).
\end{itemize}

\subsection{4. Implementación}
Asegurar pruebas UAT, ambiente controlado, versionamiento y validación final antes de producción. Para modelos ML/DL, asegurar pipeline de inferencia.

\subsection{5. Monitoreo Continuo}
\begin{itemize}
  \item Verificación de rendimiento (deterioro, drift, recalibración).
  \item Indicadores clave: pérdida de precisión, uso inapropiado, cambio de datos.
\end{itemize}

\subsection{6. Retiro o Actualización}
Debe haber criterios objetivos para desactivar o actualizar modelos obsoletos.

\section{Criterios de Validación y Monitoreo}
\begin{itemize}
  \item \textbf{Modelos Supervisados}: métricas de clasificación o regresión.
  \item \textbf{Modelos No Supervisados}: validación de estabilidad de clusters.
  \item \textbf{Modelos DL}: revisión de arquitectura, número de capas, tiempos de entrenamiento, curvas de pérdida, validación cruzada.
\end{itemize}

\section{Documentación y Trazabilidad}
Cada modelo debe contar con documentación obligatoria:
\begin{enumerate}
  \item Descripción del modelo y objetivo.
  \item Dataset utilizado y origen de los datos.
  \item Código fuente con control de versiones.
  \item Resultados de validación y backtesting.
  \item Registro de aprobaciones y uso en decisiones.
\end{enumerate}

\section{Gestión de Riesgos del Modelo}
\begin{itemize}
  \item Identificación de riesgos inherentes al tipo de modelo.
  \item Planes de contingencia y alertas tempranas.
  \item Tolerancias definidas para métricas críticas.
\end{itemize}

\section{Requerimientos Regulatorios y Auditoría}
\begin{itemize}
  \item Cumplimiento con normativas locales e internacionales (ej. IFRS 9, Basilea).
  \item Trazabilidad ante requerimientos de supervisores o auditores.
  \item Archivos históricos de decisiones, versiones y resultados.
\end{itemize}

\section{Tecnología y Herramientas Recomendadas}
Para asegurar eficiencia, trazabilidad y escalabilidad en la gestión de modelos, se recomienda el uso de las siguientes herramientas:
\begin{itemize}
  \item \textbf{Versionamiento y colaboración}: Git, GitLab, Bitbucket
  \item \textbf{Gestión de entornos y dependencias}: Docker, Conda
  \item \textbf{Automatización y orquestación}: Airflow, MLflow, Prefect
  \item \textbf{Monitoreo de desempeño}: Evidently AI, Prometheus + Grafana, Power BI
  \item \textbf{Almacenamiento seguro}: Data Lake institucional, Azure Blob Storage
  \item \textbf{Servidores de producción}: Azure Machine Learning, Amazon SageMaker, Kubernetes
\end{itemize}

\section{Cronograma de Implementación}
\begin{itemize}
  \item \textbf{Mes 1–2}: Diagnóstico del estado actual, consolidación de inventario de modelos
  \item \textbf{Mes 3–4}: Definición de políticas y formalización de roles/comités
  \item \textbf{Mes 5–6}: Diseño e implementación del flujo de ciclo de vida del modelo
  \item \textbf{Mes 7–8}: Capacitación de equipos, implementación de herramientas
  \item \textbf{Mes 9 en adelante}: Monitoreo, auditorías internas, mejora continua
\end{itemize}

\newpage
\appendix
\section*{Anexos}

\subsection*{Anexo A: Ejemplos de Métricas de Performance por Tipo de Modelo}
\begin{itemize}
  \item Clasificación: Accuracy, Precision, Recall, F1 Score, AUC-ROC
  \item Regresión: RMSE, MAE, R\textsuperscript{2}, MAPE
  \item Modelos Probabilísticos: LogLoss, Brier Score
  \item Modelos No Supervisados: Silhouette Score, Calinski-Harabasz Index
\end{itemize}

\subsection*{Anexo B: Plantilla de Inventario de Modelos}
\begin{tabular}{|l|l|l|l|l|}
\hline
\textbf{ID Modelo} & \textbf{Nombre} & \textbf{Área} & \textbf{Tipo} & \textbf{Crítico (Sí/No)} \\
\hline
001 & PD Préstamos & Riesgo Crédito & ML (XGBoost) & Sí \\
\hline
002 & Clustering Clientes & Segmentación & K-Means & No \\
\hline
\end{tabular}

\subsection*{Anexo C: Formato de Ficha Técnica del Modelo}
\begin{itemize}
  \item Nombre del Modelo: 
  \item Descripción: 
  \item Responsable del Desarrollo: 
  \item Fecha de Implementación: 
  \item Versión Actual: 
  \item Dataset Utilizado: 
  \item Algoritmo o Técnica Aplicada: 
  \item Métricas de Desempeño: 
  \item Resultados de Validación: 
  \item Frecuencia de Monitoreo: 
  \item Fecha Última Revisión: 
\end{itemize}

\subsection*{Anexo D: Ejemplo de Ficha Técnica - Modelo de PD con XGBoost}
\begin{itemize}
  \item \textbf{Nombre del Modelo:} Modelo de Probabilidad de Incumplimiento (PD) para Tarjetas de Crédito
  \item \textbf{Descripción:} Modelo supervisado que estima la probabilidad de que un cliente entre en mora de 90+ días en los próximos 12 meses, utilizando datos transaccionales, demográficos y macroeconómicos.
  \item \textbf{Responsable del Desarrollo:} Equipo de Ciencia de Datos - Subgerencia de Riesgo Crediticio
  \item \textbf{Fecha de Implementación:} 15 de marzo de 2024
  \item \textbf{Versión Actual:} v1.3
  \item \textbf{Dataset Utilizado:} Transacciones históricas 2019–2023, Buró de crédito, Segmentación geográfica
  \item \textbf{Algoritmo o Técnica Aplicada:} XGBoost (extreme gradient boosting) con validación cruzada estratificada 5-fold
  \item \textbf{Métricas de Desempeño:}
    \begin{itemize}
        \item AUC: 0.88
        \item Accuracy: 82.4\%
        \item Gini: 0.76
        \item MAPE (validación): 6.8\%
    \end{itemize}
  \item \textbf{Resultados de Validación:} Validado por el área de Riesgo Modelos el 20 de marzo de 2024. Se detectó leve sensibilidad a la variable macroeconómica IPC, controlada con escalamiento mensual.
  \item \textbf{Frecuencia de Monitoreo:} Mensual (monitoreo de drift y desempeño)
  \item \textbf{Fecha Última Revisión:} 5 de junio de 2025
\end{itemize}


\end{document}


