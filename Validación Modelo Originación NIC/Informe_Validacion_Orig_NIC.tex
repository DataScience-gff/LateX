\documentclass[12pt]{article}

% ===== Configuración de página =====
\usepackage[a4paper,margin=25mm]{geometry}
\usepackage[utf8]{inputenc}

% ===== Paquetes visuales =====
\usepackage{eso-pic}
\usepackage{tikz}
\usetikzlibrary{calc}
\usepackage{xcolor}
\usepackage{graphicx}
\usepackage{lipsum}

% ===== Idioma y fecha en español =====
\usepackage[spanish,es-lcroman]{babel}
\usepackage{datetime}
\renewcommand{\dateseparator}{de }
\renewcommand{\today}{\number\day~de~\ifcase\month\or
enero\or febrero\or marzo\or abril\or mayo\or junio\or
julio\or agosto\or septiembre\or octubre\or noviembre\or diciembre\fi~de~\number\year}

% ===== Paleta institucional =====
\definecolor{FicohsaBlue}{HTML}{002B6B}
\definecolor{FicohsaCyan}{HTML}{00B5E2}
\definecolor{FicohsaGray}{HTML}{7A8892}

% ===== Parámetros EDITABLES =====
\newcommand{\FICOLOGOFILE}{Ficohsa.png} % <-- CAMBIA ESTA RUTA
\newcommand{\FICOLOGOWIDTH}{22mm}
\newcommand{\LogoOffsetX}{12mm}
\newcommand{\LogoOffsetY}{6mm}
\newcommand{\BarWidth}{10mm}

% ===== Fondo persistente (todas las páginas) =====
\AddToShipoutPictureBG{%
  \begin{tikzpicture}[remember picture,overlay]
    % Barra azul vertical
    \fill[FicohsaBlue]
      (current page.south west) rectangle
      ($ (current page.north west) + (\BarWidth,0) $);
  \end{tikzpicture}%
}

% ===== Logo persistente (todas las páginas) =====
\AddToShipoutPictureFG{%
  \begin{tikzpicture}[remember picture,overlay]
    \node[anchor=north west,inner sep=0pt] at
      ($ (current page.north west) + (\LogoOffsetX,-\LogoOffsetY) $)
    {%
      \IfFileExists{\FICOLOGOFILE}{%
        \includegraphics[width=\FICOLOGOWIDTH]{\FICOLOGOFILE}%
      }{%
        \fbox{\bfseries \large FICOHSA}%
      }%
    };
  \end{tikzpicture}%
}

% ===== Índice =====
\setcounter{tocdepth}{2}
\setcounter{secnumdepth}{2}

% ==================== DOCUMENTO ====================
\begin{document}

% ======== PORTADA =========
\thispagestyle{empty}

% Fondo decorativo adicional de portada (líneas suaves)
\begin{tikzpicture}[remember picture,overlay]
  \draw[FicohsaCyan!40, line width=10pt, line cap=round]
    ($ (current page.south east) + (-5cm,1.5cm) $)
    to[out=180,in=-10]
    ($ (current page.south west) + (5cm,2cm) $);

  \draw[FicohsaCyan!60, line width=4pt, line cap=round]
    ($ (current page.south east) + (-6cm,3cm) $)
    to[out=190,in=0]
    ($ (current page.south west) + (3cm,2.5cm) $);
\end{tikzpicture}

\vspace*{5cm}
\begin{center}
{\Huge\bfseries Informe de Validación Modelo de Originación Nicaragua\par}
\vspace{1cm}
{\Large Subgerencia de Ciencia de Datos –  Riesgos\par}
\vspace{0.3cm}
{\large Grupo Financiero Ficohsa\par}
\vspace{3cm}
{\Large \textcolor{FicohsaBlue}{\textbf{Confidencial}}\par}
\vfill
{\large Tegucigalpa, Honduras – \today}
\end{center}

\newpage

% ===== ÍNDICE =====
\tableofcontents
\newpage

% ===== CONTENIDO =====

%------------------------------------- Introducción -------------------------------------
\section{Objetivo de la validación}

Evaluar la solidez metodológica, calidad de datos y capacidad predictiva del modelo de originaciones de tarjeta de crédito desarrollado por Banco Ficohsa Nicaragua;  verificar coherencia  del scorecard, backtesting/estabilidad y preparación para producción, conforme a las mejores prácticas internas (Gobernanza de Modelo) y de industria. Nos enfocamos en evidencia y trazabilidad de lo reportado por el desarrollador.  Puntualmente buscamos asegurar que el modelo: 

\begin{itemize}
\item Cumpla los estándares definidos por el Marco de Gobernanza de Modelos desarrollada por Riesgos.
\item Se encuentre alineado con las mejores prácticas regulatorias y de la industria en materia de riesgos. 
\item Presente resultados consistentes, transparentes y trazables, garantizando una implementación segura y robusta en producción. 
\end{itemize}

\textbf{\underline{Nota Aclaratoria:}} El presente informe se enfoca en validar el modelo y señalar aspectos \textbf{no documentados en el reporte técnico del desarrollador}, que son críticos para su implementación segura y robusta en producción. 

% ------------------------------------ Alcance Validación --------------------------------
\section{Alcance}

La presente validación abarca la revisión del modelo de originaciones de crédito desarrollado por Banco Ficohsa Honduras. El alcance incluye los siguientes elementos: 
\begin{itemize}
\item \textbf{Revisión Documental:} análisis del informe técnico del desarrollador, verificando la completitud y consistencia de la información presentada. 
\item \textbf{Calidad de Datos:} evaluación de la completitud, consistencia y trazabilidad de las fuentes de datos utilizadas en el modelo. 
\item \textbf{Procesamiento de Variables:} revisión de los procedimientos de depuración, imputación, tratamiento de outliers, control de multicolinealidad y selección de variables. 
\item \textbf{Metodología de Modelado:} análisis de la solidez estadística de los enfoques aplicados, criterios de selección de la técnica final. 
\item \textbf{Desempeño Predictivo:} Revisión de las métricas de discriminación (KS, AUC, GINI), precisión y estabilidad temporal del modelo. \textbf{Backtesting y estabilidad:} revisión de las pruebas fuera de muestra, así como de las métricas de monitoreo propuestas.
\end{itemize}
\textbf{\underline{Exclusiones del Alcance}}
\begin{itemize}
\item No se cuantifica impactos financieros directos (pricing, rentabilidad ajustada al riesgo, etc.).
\item La validación no sustituye la responsabilidad del área desarrolladora en cuanto a la correcta ejecución técnica del código. 
\end{itemize}
\newpage

% ----------------------------------- Resumen del Modelo a Validar ----------------------------------
\subsection{Resumen del Modelo a Validar}

El propósito de este modelo es predecir la probabilidad de que un cliente incurra en mora $\geq 90$ días (M90+) dentro de los primeros 12 meses posteriores a la activación del producto de tarjeta de crédito. El objetivo responde al patrón histórico de siniestralidad, que alcanza su máximo entre 6 y 12 meses de madurez.

\begin{itemize}
\item \textbf{Tipo de Modelo:} Regresión Logística, por su interpretabilidad e integración natural a scorecard; seleccion de variables con \textbf{RFECV} usando \textbf{XGBoost} como estimador base, \textbf{validación cruzada 10-fold} y \textbf{optimización con Optuna}.
\item \textbf{Variable Objetivo:} cliente \textbf{malo} si alcanza M90+ en los 12 meses posteriores a su originación; \textbf{bueno} en caso contrario (definiciones explícitas de goods/bads). 
\item \textbf{Periodo de Datos:} 
\begin{itemize}
\item Observación: cosechas de mayo 2023 - abril 2024. 
\item Desempeño: mayo 2024 - abril 2025 (12 meses).
\end{itemize}
\item \textbf{Fuentes de Datos:}\\ DBDWHNG (Oracle) $\longrightarrow$ CF\_NI\_ORIG $\longrightarrow$ T\_CLIT\_ANALISISSOLICITUDCREDIT (solicitudes; integra burós y datos demográficos). Los desarrolladores trabajaron exclusivamente con esta fuente interna en Nicaragua. 
\item \textbf{Población analizada:} 
\begin{itemize}
\item \textbf{Inicial externa:} 127,399 primeras observaciones; tras reglas de elegibilidad y "no-hit", población final con \textbf{8,721 buenos (80.9\%)} y \textbf{2,056 malos (9.1\%)}. 
\item \textbf{Inicial Interna:} 21,412 exclusiones operativas (corporativas/PYMES, segundas cuentas, migradas, colaboradores, pignoradas, sin uso, etc.) $\longrightarrow$ \textbf{7,706 buenos (90.8\%)} y \textbf{780 malos (9.19\%)} 
\end{itemize}
\textbf{Nota:} el documento NIC también resume una \textbf{población total} (inicial 148,811; final 19,263) en su resumen ejecutivo.
\end{itemize}
%\section{Metodología}
%\subsection{Datos y preparación}
%\lipsum[5]
%\subsection{Modelado}
%\lipsum[6]

%\section{Resultados}
%\subsection{Métricas}
%\lipsum[7]
%\subsection{Análisis}
%\lipsum[8]

%\section{Conclusiones y Recomendaciones}
%\lipsum[9]

\end{document}


 