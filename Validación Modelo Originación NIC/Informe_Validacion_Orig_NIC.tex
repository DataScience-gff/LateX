\documentclass[12pt]{article}

% ===== Configuración de página =====
\usepackage[a4paper,margin=25mm]{geometry}
\usepackage[utf8]{inputenc}

% ===== Paquetes visuales =====
\usepackage{eso-pic}
\usepackage{tikz}
\usetikzlibrary{calc}
\usepackage{xcolor}
\usepackage{graphicx}
\usepackage{lipsum}
\usepackage{float}

\usepackage[table,xcdraw]{xcolor}
\usepackage{array}
\usepackage{caption}
\captionsetup[table]{skip=5pt,labelfont=bf,font=small}
\usepackage[table,xcdraw]{xcolor}
\usepackage{booktabs}
\definecolor{FicohsaBlue}{HTML}{0033A0}

\usepackage[table,xcdraw]{xcolor}
\usepackage{booktabs}
\usepackage{float}
\usepackage{graphicx} % para \resizebox
\definecolor{FicohsaBlue}{HTML}{0033A0}

\usepackage{tcolorbox}
\tcbset{colback=gray!5,colframe=FicohsaBlue!85,arc=2mm,boxrule=0.6pt,left=6pt,right=6pt,top=4pt,bottom=4pt}


% ===== Idioma y fecha en español =====
\usepackage[spanish,es-lcroman]{babel}
\usepackage{datetime}
\renewcommand{\dateseparator}{de }
\renewcommand{\today}{\number\day~de~\ifcase\month\or
enero\or febrero\or marzo\or abril\or mayo\or junio\or
julio\or agosto\or septiembre\or octubre\or noviembre\or diciembre\fi~de~\number\year}

% ===== Paleta institucional =====
\definecolor{FicohsaBlue}{HTML}{002B6B}
\definecolor{FicohsaCyan}{HTML}{00B5E2}
\definecolor{FicohsaGray}{HTML}{7A8892}

% ===== Parámetros EDITABLES =====
\newcommand{\FICOLOGOFILE}{Ficohsa.png} % <-- CAMBIA ESTA RUTA
\newcommand{\FICOLOGOWIDTH}{22mm}
\newcommand{\LogoOffsetX}{12mm}
\newcommand{\LogoOffsetY}{6mm}
\newcommand{\BarWidth}{10mm}

% ===== Fondo persistente (todas las páginas) =====
\AddToShipoutPictureBG{%
  \begin{tikzpicture}[remember picture,overlay]
    % Barra azul vertical
    \fill[FicohsaBlue]
      (current page.south west) rectangle
      ($ (current page.north west) + (\BarWidth,0) $);
  \end{tikzpicture}%
}

% ===== Logo persistente (todas las páginas) =====
\AddToShipoutPictureFG{%
  \begin{tikzpicture}[remember picture,overlay]
    \node[anchor=north west,inner sep=0pt] at
      ($ (current page.north west) + (\LogoOffsetX,-\LogoOffsetY) $)
    {%
      \IfFileExists{\FICOLOGOFILE}{%
        \includegraphics[width=\FICOLOGOWIDTH]{\FICOLOGOFILE}%
      }{%
        \fbox{\bfseries \large FICOHSA}%
      }%
    };
  \end{tikzpicture}%
}

% ===== Índice =====
\setcounter{tocdepth}{2}
\setcounter{secnumdepth}{2}

% ==================== DOCUMENTO ====================
\begin{document}

% ======== PORTADA =========
\thispagestyle{empty}

% Fondo decorativo adicional de portada (líneas suaves)
\begin{tikzpicture}[remember picture,overlay]
  \draw[FicohsaCyan!40, line width=10pt, line cap=round]
    ($ (current page.south east) + (-5cm,1.5cm) $)
    to[out=180,in=-10]
    ($ (current page.south west) + (5cm,2cm) $);

  \draw[FicohsaCyan!60, line width=4pt, line cap=round]
    ($ (current page.south east) + (-6cm,3cm) $)
    to[out=190,in=0]
    ($ (current page.south west) + (3cm,2.5cm) $);
\end{tikzpicture}

\vspace*{5cm}
\begin{center}
{\Huge\bfseries Informe de Validación Modelo de Originación Nicaragua\par}
\vspace{1cm}
{\Large Subgerencia de Ciencia de Datos –  Riesgos\par}
\vspace{0.3cm}
{\large Grupo Financiero Ficohsa\par}
\vspace{3cm}
{\Large \textcolor{FicohsaBlue}{\textbf{Confidencial}}\par}
\vfill
{\large Tegucigalpa, Honduras – \today}
\end{center}

\newpage

% ===== ÍNDICE =====
\tableofcontents
\newpage

% ===== CONTENIDO =====

%------------------------------------- Introducción -------------------------------------
\section{Objetivo de la validación}

Evaluar la solidez metodológica, calidad de datos y capacidad predictiva del modelo de originaciones de tarjeta de crédito desarrollado por Banco Ficohsa Nicaragua;  verificar coherencia  del scorecard, backtesting/estabilidad y preparación para producción, conforme a las mejores prácticas internas (Gobernanza de Modelo) y de industria. Nos enfocamos en evidencia y trazabilidad de lo reportado por el desarrollador.  Puntualmente buscamos asegurar que el modelo: 

\begin{itemize}
\item Cumpla los estándares definidos por el Marco de Gobernanza de Modelos desarrollada por Riesgos.
\item Se encuentre alineado con las mejores prácticas regulatorias y de la industria en materia de riesgos. 
\item Presente resultados consistentes, transparentes y trazables, garantizando una implementación segura y robusta en producción. 
\end{itemize}

\textbf{\underline{Nota Aclaratoria:}} El presente informe se enfoca en validar el modelo y señalar aspectos \textbf{no documentados en el reporte técnico del desarrollador}, que son críticos para su implementación segura y robusta en producción. 

% ------------------------------------ Alcance Validación --------------------------------
\section{Alcance}

La presente validación abarca la revisión del modelo de originaciones de crédito desarrollado por Banco Ficohsa Nicaragua. El alcance incluye los siguientes elementos: 
\begin{itemize}
\item \textbf{Revisión Documental:} análisis del informe técnico del desarrollador, verificando la completitud y consistencia de la información presentada. 
\item \textbf{Calidad de Datos:} evaluación de la completitud, consistencia y trazabilidad de las fuentes de datos utilizadas en el modelo. 
\item \textbf{Procesamiento de Variables:} revisión de los procedimientos de depuración, imputación, tratamiento de outliers, control de multicolinealidad y selección de variables. 
\item \textbf{Metodología de Modelado:} análisis de la solidez estadística de los enfoques aplicados, criterios de selección de la técnica final. 
\item \textbf{Desempeño Predictivo:} Revisión de las métricas de discriminación (KS, AUC, GINI), precisión y estabilidad temporal del modelo. \textbf{Backtesting y estabilidad:} revisión de las pruebas fuera de muestra, así como de las métricas de monitoreo propuestas.
\end{itemize}
\textbf{\underline{Exclusiones del Alcance}}
\begin{itemize}
\item No se cuantifica impactos financieros directos (pricing, rentabilidad ajustada al riesgo, etc.).
\item La validación no sustituye la responsabilidad del área desarrolladora en cuanto a la correcta ejecución técnica del código. 
\end{itemize}
\newpage

% ----------------------------------- Resumen del Modelo a Validar ----------------------------------
\section{Resumen del Modelo a Validar}

Este modelo corresponde a la versión 2025 del score de originación de tarjetas de créditos, desarrollado por el equipo de Analítica de Crédito Nicaragua.\\

El propósito de este modelo es predecir la probabilidad de que un cliente incurra en mora $\geq 90$ días (M90+) dentro de los primeros 12 meses posteriores a la activación del producto de tarjeta de crédito. El objetivo responde al patrón histórico de siniestralidad, que alcanza su máximo entre 6 y 12 meses de madurez.

\begin{itemize}
\item \textbf{Tipo de Modelo:} Regresión Logística, por su interpretabilidad e integración natural a scorecard; seleccion de variables con \textbf{RFECV} usando \textbf{XGBoost} como estimador base, \textbf{validación cruzada 10-fold} y \textbf{optimización con Optuna}.
\item \textbf{Variable Objetivo:} cliente \textbf{malo} si alcanza M90+ en los 12 meses posteriores a su originación; \textbf{bueno} en caso contrario (definiciones explícitas de goods/bads). 
\item \textbf{Periodo de Datos:} 
\begin{itemize}
\item Observación: cosechas de mayo 2023 - abril 2024. 
\item Desempeño: mayo 2024 - abril 2025 (12 meses).
\end{itemize}
\item \textbf{Fuentes de Datos:}\\ DBDWHNG (Oracle) $\longrightarrow$ CF\_NI\_ORIG $\longrightarrow$ T\_CLIT\_ANALISISSOLICITUDCREDIT (solicitudes; integra burós y datos demográficos). Los desarrolladores trabajaron exclusivamente con esta fuente interna en Nicaragua. 
\item \textbf{Población analizada:} 
\begin{itemize}
\item \textbf{Inicial externa:} 127,399 primeras observaciones; tras reglas de elegibilidad y "no-hit", población final con \textbf{8,721 buenos (80.9\%)} y \textbf{2,056 malos (9.1\%)}. 
\item \textbf{Inicial Interna:} 21,412 exclusiones operativas (corporativas/PYMES, segundas cuentas, migradas, colaboradores, pignoradas, sin uso, etc.) $\longrightarrow$ \textbf{7,706 buenos (90.8\%)} y \textbf{780 malos (9.19\%)} 
\end{itemize}
\textbf{Nota:} el documento NIC también resume una \textbf{población total} (inicial 148,811; final 19,263) en su resumen ejecutivo.
\item \textbf{Variables explicativas seleccionadas:} señales de riesgo/experiencia y capacidad provenientes de TUCA/SIBOIF:
\begin{itemize}
\item FL\_MORA\_ACT\_TUCA
\item  FL\_MORA\_90\_24M
\item CANT\_DIAS\_MORA\_ATC
\item CANT\_INST\_SIBOIF
\item UTILIZACION\_TC\_SIBOIF
\item PCT\_INGRESO\_CUOTA\_TUCA
\item CANT\_EXP\_VIGENTE\_TUCA
\item CANT\_EXP\_HIST\_SIBOIF
\end{itemize}
\item \textbf{Scorecard y escalamiento:} Score mínimo 201 / máximo 293 con scaling 10:1\@300 y PDO=10; se documentan tablas de puntos por variables (incluyendo manejo de "SIN\_INFORMACION").

\item \textbf{Métricas clave (dev/test/OOT):}  
\begin{itemize}
\item KS (desarrollo): \fbox{0.34}
\item AUC: \fbox{0.78}
\item KS (validación fuera de muestra/tiempo OOT): \fbox{41.66\%} (mayo 2024 - abril 2025). 
\end{itemize}
\item \textbf{Corte operativo sugerido:} 261 puntos, con tasa de malos esperada 12.5\% y tasa de aprobación 89 - 89.3\%; se muestran curvas KS y zonas de aceptación/rechazo.
\item \textbf{Comparativo con modelo anterior (benchmark):} Este modelo supera el desempeño del anterior (documento reporta mejora de KS:34\% vs 25\% en referencia histórica y 41.66\% OOT en la ventana oficial). 
\end{itemize}



\begin{table}[H]
{\small 
\centering
\renewcommand{\arraystretch}{1.3}
\rowcolors{2}{gray!5}{white}
\setlength{\tabcolsep}{8pt}
\label{tab:resumen_modelo_nic}
\begin{tabular}{|p{5cm}|p{10cm}|}
\rowcolor{FicohsaBlue!85}
\multicolumn{2}{|c|}{\color{white}\textbf{Resumen Ejecutivo del Modelo}} \\ \hline
\textbf{Elemento} & \textbf{Detalle principal} \\ \hline
Modelo & Regresión Logística (Optuna + RFECV con XGBoost como estimador base) \\ \hline
Variable objetivo & Mora $\geq$ 90 días dentro de los 12 meses posteriores a la originación \\ \hline
Ventana de observación & Mayo 2023 – Abril 2024 \\ \hline
Ventana de desempeño (OOT) & Mayo 2024 – Abril 2025 \\ \hline
Fuente de datos & \texttt{DBDWHNG (Oracle) → CF\_NI\_ORIG → T\_CLIT\_ANALISISSOLICITUDCREDIT} \\ \hline
Población final (interna + externa) & 19,263 registros válidos \\ \hline
KS (desarrollo) & 0.34 \\ \hline
AUC & 0.78 \\ \hline
KS (validación fuera de muestra – OOT) & 0.4166 \\ \hline
Corte operativo sugerido & 261 puntos \\ \hline
Tasa de malos esperada & 12.5\% \\ \hline
Tasa de aprobación esperada & 89.0–89.3\% \\ \hline
Comparativo con modelo anterior & Mejora de KS: 34\% vs 25\% (histórico) y 41.66\% OOT en ventana oficial \\ \hline
\end{tabular}
}
\end{table}

El modelo presenta un desempeño sólido (KS = 0.34 en desarrollo; KS = 0.4166 OOT; AUC = 0.78), evidenciando mejora significativa respecto al modelo vigente y cumplimiento con los estandares de discriminación exigidos por la Gobernanza de Modelos de Grupo Financiero Ficohsa. 

\section{Análisis del Validador}
\subsection{Calidad y Preparación de Datos}

El modelo utiliza como fuente DBDWHNG (Oracle), esquema CF\_NI\_ORIG y tabla T\_CLIT\_ANALISISSOLICITUDCREDIT, 	integrando información de burós TUCA/SIBOIF y datos demográficos. Se documentan las cascadas internas y externas y las exclusiones operativas de población. 

\subsubsection{\underline{Hallazgos y vacios identificados}}

\subsubsection*{Completitud de datos (Cr\'itico)}
No se presenta una \textbf{tabla de cobertura por variable} (train/test/OOT) que detalle los porcentajes de \textbf{valores faltantes}, \textbf{ceros} y \textbf{cardinalidad}, ni se muestran las estad\'isticas descriptivas b\'asicas (media, percentiles). 
Esta informaci\'on es indispensable para evaluar la calidad y estabilidad de los insumos del modelo y garantizar que no existan variables con vac\'ios o sesgos estructurales.\\

\textbf{Acci\'on requerida:} Incluir una matriz de completitud que refleje la calidad de cada variable utilizada, considerando los tres per\'iodos (train, test y OOT), e incorporar las principales estad\'isticas descriptivas: media, P5, P50, P95 y P99.

\vspace{0.5em}
\renewcommand{\arraystretch}{1.2}
\rowcolors{2}{gray!5}{white}
Tomen como ejemplo ilustrativo la siguiente tabla:
\begin{table}[H]
\centering
\scriptsize % Reduce tamaño de fuente
\setlength{\tabcolsep}{4pt} % Reduce espacio entre columnas
\renewcommand{\arraystretch}{1.2}
\label{tab:completitud_ejemplo}
\begin{tabular}{p{3.2cm}rrrrrrr} % Reduce el ancho de la primera columna
\rowcolor{FicohsaBlue!85}
\color{white}\textbf{Variable} &
\color{white}\textbf{\% Missing (Train)} &
\color{white}\textbf{\% Missing (OOT)} &
\color{white}\textbf{\% Ceros} &
\color{white}\textbf{Card.} &
\color{white}\textbf{Media} &
\color{white}\textbf{P50} &
\color{white}\textbf{P99} \\
\hline
{\tiny UTILIZACION\_TC\_SIBOIF} & 0.0\% & 0.2\% & 1.3\% & 101 & 0.37 & 0.36 & 0.98 \\
{\tiny PCT\_INGRESO\_CUOTA\_TUCA} & 1.1\% & 1.5\% & 0.0\% & 750 & 0.28 & 0.27 & 0.89 \\
{\tiny CANT\_DIAS\_MORA\_ATC} & 0.0\% & 0.0\% & 67.5\% & 365 & 3.9 & 0 & 60 \\
{\tiny FL\_MORA\_90\_24M} & 0.0\% & 0.0\% & 0.0\% & 2 & 0.09 & 0 & 1 \\
\hline
\end{tabular}
\end{table}

\subsubsection*{Linaje de variables (Cr\'itico)}
No se presenta una \textbf{bit\'acora de transformaciones} que vincule cada campo fuente con su variable modelada (origen $\rightarrow$ transformaci\'on $\rightarrow$ modelo/score). 
Sin esta trazabilidad, no es posible \textit{auditar}, \textit{reproducir} ni \textit{operacionalizar} el modelo de forma segura.\\

\textbf{Acci\'on requerida:} Incorporar un \textbf{glosario de linaje} que detalle, para cada variable del modelo: 
(i) campo(s) fuente; (ii) transformaciones aplicadas (winsor, log, escalas, ratios, bins/WOE, flags de ``SIN INFO''); 
(iii) variable final usada en el modelo/score; (iv) versi\'on y fecha de la regla vigente.

\vspace{2em}

\textbf{Consideren el siguiente ejemplo ilustrativo de glosario de linaje (origen $\rightarrow$ transformaci\'on $\rightarrow$ modelo/score):}
%===========================
% TABLA 1 - LINAJE DE VARIABLES
%===========================
\renewcommand{\arraystretch}{1.15}
\begin{table}[H]
\centering
\label{tab:linaje_ejemplo}
\resizebox{\textwidth}{!}{%
\begin{tabular}{p{3.7cm} p{6.0cm} p{4.3cm} p{2.2cm}}
\rowcolor{FicohsaBlue!85}
\color{white}\textbf{Campo(s) fuente (DB/bur\'o)} 
& \color{white}\textbf{Transformaci\'on aplicada} 
& \color{white}\textbf{Variable en modelo} 
& \color{white}\textbf{Versi\'on } \\
\toprule
\texttt{TUCA\_CUOTA\_MENSUAL}, \texttt{INGRESO\_DECLARADO} 
& Ratio \texttt{CUOTA/INGRESO}; winsor P99 por canal; imputaci\'on mediana por canal; flag \texttt{SIN\_INFO} 
& \texttt{PCT\_INGRESO\_CUOTA\_TUCA} 
& v1.0 / 2024-09 \\
\midrule
\texttt{SIBOIF\_LIM\_TC\_TOT} 
& Winsor P[1,99]; log($x{+}1$) para estabilizar; normalizaci\'on min-max; flag \texttt{SIN\_INFO} 
& \texttt{TOTAL\_LINEA\_TC\_SIBOIF} 
& v1.0 / 2024-09 \\
\midrule
\texttt{TUCA\_MORA\_90\_24M} 
& Binario (1=presencia en 24m); imputaci\'on 0 si \texttt{NO\_HIT}; binning monot\'onico $\rightarrow$ WOE 
& \texttt{FL\_MORA\_90\_24M} 
& v1.0 / 2024-09 \\
\midrule
\texttt{TUCA\_UTILIZACION\_TC} 
& Promedio 3m; winsor P[1,99]; reescala 0--1; sustituci\'on por mediana global si \texttt{NO\_HIT} 
& \texttt{UTILIZACION\_TC\_SIBOIF} 
& v1.0 / 2024-09 \\
\midrule
\texttt{TUCA\_DIAS\_MORA\_ACT} 
& Top-coding a 60 d\'ias; binning supervisado; mapeo a puntos en scorecard 
& \texttt{CANT\_DIAS\_MORA\_ATC} 
& v1.0 / 2024-09 \\
\bottomrule
\end{tabular}
}
\end{table}

\noindent\textit{Nota:} El glosario de linaje debe cubrir \textbf{todas} las variables del modelo, incluyendo las descartadas (con justificaci\'on), y mantener control de \textbf{versiones} y \textbf{fechas de vigencia} por gobernanza.

\vspace{2em}


\noindent\textbf{Criterio del validador.} Con estas evidencias, el linaje queda \textit{auditado} extremo a extremo y se mitiga el riesgo de \textit{fuga de informaci\'on} y de \textit{no reproducibilidad} en producci\'on. Adem\'as, la trazabilidad respalda diferencias observadas en desempe\~no OOT (p.ej., $KS_{OOT}$ $>$ $KS_{Train}$) al conectar mezcla, latencias y reglas de transformaci\'on vigentes.


\subsubsection*{Tratamiento de outliers e imputaciones (Alto)}
No se documentan los \textbf{criterios de winsorizaci\'on} ni las \textbf{reglas de imputaci\'on} aplicadas a las variables del modelo. 
Tampoco se especifica el \textbf{porcentaje de registros afectados} por cada tratamiento ni la metodolog\'ia empleada (media, mediana, WOE o segmentada por canal).\\

Este componente es cr\'itico para garantizar la \textbf{estabilidad de los coeficientes log\'isticos} y evitar que valores extremos o faltantes distorsionen el score o el ranking de riesgo.\\

\textbf{Acci\'on requerida:} El desarrollador debe entregar un inventario de variables afectadas por outliers e imputaciones, detallando: 
(i) m\'etodo aplicado, (ii) porcentaje de registros modificados, (iii) regla de reemplazo y (iv) validaci\'on post-tratamiento.

\vspace{2em}

\textbf{Ejemplo ilustrativo de tratamiento de outliers e imputaciones}

\renewcommand{\arraystretch}{1.15}
\begin{table}[H]
\centering
\label{tab:outliers_imputaciones}
\resizebox{\textwidth}{!}{%
\begin{tabular}{p{4.0cm} p{5.8cm} r p{5.8cm}}
\rowcolor{FicohsaBlue!85}
\color{white}\textbf{Variable} &
\color{white}\textbf{Criterio de tratamiento} &
\color{white}\textbf{\% Registros afectados} &
\color{white}\textbf{Estrategia de reemplazo} \\
\toprule
\texttt{UTILIZACION\_TC\_SIBOIF} &
Winsorizaci\'on en percentiles P[1,99]; detecci\'on de valores extremos por IQR. &
1.7\% &
Reescalado al percentil 99 (0.98); valores faltantes reemplazados por mediana global. \\
\midrule
\texttt{PCT\_INGRESO\_CUOTA\_TUCA} &
Outliers superiores P99 truncados; imputaci\'on segmentada por canal. &
0.9\% &
Valores extremos reemplazados por media por canal; \texttt{NO\_HIT} imputado con mediana general. \\
\midrule
\texttt{CANT\_DIAS\_MORA\_ATC} &
Top-coding a 60 d\'ias; valores negativos tratados como 0. &
2.4\% &
Cap a 60; validaci\'on posterior muestra mejora en estabilidad (AUC +0.6pp). \\
\midrule
\texttt{TOTAL\_LINEA\_TC\_SIBOIF} &
Winsor P[1,99]; log-transform para reducir asimetr\'ia; imputaci\'on con ratio global. &
1.2\% &
Aplicaci\'on de log(x+1); mejora de normalidad (skewness -0.9). \\
\bottomrule
\end{tabular}
}
\end{table}

\noindent\textit{Nota:} La documentaci\'on de tratamientos de outliers e imputaciones permite asegurar la coherencia de los datos de entrada y la estabilidad del modelo. 
Cada variable debe contar con una ficha de control que detalle el criterio aplicado, el impacto porcentual y la evidencia de mejora en la estabilidad del score o del AUC post-tratamiento.

\subsubsection*{Estabilidad de insumos (Cr\'itico)}

No se incluye el \textbf{PSI (Population Stability Index)} por variable ni el \textbf{PSI global} entre los per\'iodos de desarrollo y OOT. 
Este an\'alisis es fundamental para evaluar la \textbf{estabilidad temporal de las distribuciones} y explicar las diferencias en desempe\~no entre ventanas, 
especialmente considerando que el \textbf{KS OOT (0.4166)} supera al \textbf{KS Train (0.34)}.\\

\textbf{Acci\'on requerida:} Calcular el PSI para cada variable y el PSI global del modelo:
\begin{itemize}
  \item \textbf{PSI $<$ 0.10} → Estable (sin cambio relevante)
  \item \textbf{$0.10 \leq$ PSI $< 0.25$} → Atenci\'on (monitorizar variable)
  \item \textbf{PSI $\geq 0.25$} → Alerta (posible drift o cambio de mezcla)
\end{itemize}

\vspace{2em}

\textbf{Ejemplo ilustrativo de estabilidad de insumos – Cálculo de PSI (Train \& OOT)}
\renewcommand{\arraystretch}{1.15}
\begin{table}[H]
\centering
\label{tab:psi_ejemplo}
\resizebox{\textwidth}{!}{%
\begin{tabular}{p{4.5cm} r r r p{4.5cm}}
\rowcolor{FicohsaBlue!85}
\color{white}\textbf{Variable} &
\color{white}\textbf{PSI} &
\color{white}\textbf{Categor\'ia} &
\color{white}\textbf{Cambio observado} &
\color{white}\textbf{Acci\'on recomendada} \\
\toprule
\texttt{UTILIZACION\_TC\_SIBOIF} & 0.07 & Estable & Distribuci\'on sin cambio significativo. & Sin acci\'on; variable estable entre ventanas. \\
\midrule
\texttt{PCT\_INGRESO\_CUOTA\_TUCA} & 0.14 & Atenci\'on & Incremento en bins altos (clientes con mayor carga). & Monitorizar; revisar cambios de mezcla en canal retail. \\
\midrule
\texttt{CANT\_DIAS\_MORA\_ATC} & 0.28 & Alerta & Mayor proporci\'on de clientes sin mora en OOT. & Revisar calidad de reporte TUCA y posible sesgo temporal. \\
\midrule
\texttt{TOTAL\_LINEA\_TC\_SIBOIF} & 0.09 & Estable & Sin cambios en la distribuci\'on de l\'imites. & Sin acci\'on. \\
\midrule
\textbf{PSI Global (modelo)} & \textbf{0.16} & \textbf{Atenci\'on} & & Cambios leves de mezcla explican mejora de KS OOT. \\
\bottomrule
\end{tabular}
}
\end{table}

\noindent\textit{Nota:} 
El PSI global de 0.16 indica una variaci\'on moderada entre las distribuciones de desarrollo y OOT. 
Las variables con PSI $>$ 0.25 deben ser revisadas por posibles cambios de mezcla o sesgos de informaci\'on (p.ej., cobertura TUCA/SIBOIF o endurecimiento de pol\'iticas). 
Este an\'alisis permitir\'a justificar de forma t\'ecnica la diferencia positiva de KS en OOT y definir umbrales de monitoreo mensual en producci\'on.\\

\vspace{0.8em}
\begin{tcolorbox}
\textbf{Conclusi\'on del Validador – Calidad y Preparaci\'on de Datos}\\[3pt]
El modelo cuenta con una estructura de datos adecuada y fuentes bien identificadas; sin embargo, la documentaci\'on de los procesos de completitud, linaje, imputaci\'on (si aplica) y estabilidad debe fortalecerse con evidencia cuantitativa. 
La inclusi\'on de estos elementos en la documentaci\'on permitir\'a garantizar la trazabilidad y estabilidad del modelo en el tiempo, asegurando que su desempe\~no fuera de muestra sea sostenible y t\'ecnicamente justificable.
\end{tcolorbox}
%---------------------------------------------------------------------------------------
\subsection{Especificaci\'on del Modelo y Significancia de Variables }

El modelo fue ajustado mediante un proceso de selección (Optuna + RFECV con XGBoost como estimador base), priorizando variables provinientes de TUCA, SIBOIF y comportamiento interno. \\

El documento lista los predictores finales y menciona la exclusión de variables con alta correlación, aunque \textbf{no presenta evidencia cuantitativa} (ni matriz de correlación, ni valores de IV o VIF).\\

\textbf{Hallazgo:} El informe t\'ecnico no presenta evidencia cuantitativa de la selecci\'on y validez de las variables. 
No se incluyen los valores de \textit{Information Value (IV)}, ni los indicadores de colinealidad (VIF) ni las pruebas de significancia estad\'istica de los coeficientes.\\

\textbf{Acci\'on requerida:} Incorporar las siguientes evidencias: tabla de IV, matriz de correlaci\'on, VIF, y estimaci\'on log\'istica con errores est\'andar, p-value y Odds Ratio (OR).

\vspace{0.8em}
\textbf{Ejemplo ilustrativo – Diagn\'ostico de Significancia y Colinealidad:}
\renewcommand{\arraystretch}{1.2}
\begin{table}[H]
\centering
\label{tab:iv_vif_logit}
\resizebox{\textwidth}{!}{%
\begin{tabular}{p{4cm} r r r r r}
\rowcolor{FicohsaBlue!85}
\color{white}\textbf{Variable} &
\color{white}\textbf{IV} &
\color{white}\textbf{VIF} &
\color{white}\textbf{Coef. Logit} &
\color{white}\textbf{p-valor} &
\color{white}\textbf{OR (IC95\%)} \\
\toprule
\texttt{UTILIZACION\_TC\_SIBOIF} & 0.24 & 2.1 & 0.85 & 0.000 & 2.35 [2.06–2.67] \\
\midrule
\texttt{PCT\_INGRESO\_CUOTA\_TUCA} & 0.18 & 3.4 & 0.41 & 0.002 & 1.51 [1.38–1.66] \\
\midrule
\texttt{CANT\_DIAS\_MORA\_ATC} & 0.31 & 1.9 & 0.33 & 0.005 & 1.39 [1.11–1.65] \\
\midrule
\texttt{TOTAL\_LINEA\_TC\_SIBOIF} & 0.10 & 2.7 & -0.22 & 0.041 & 0.80 [0.64–0.97] \\
\bottomrule
\end{tabular}
}
\end{table}

\noindent\textit{Nota:} Las variables muestran un Information Value entre 0.10 y 0.31, lo que indica capacidad predictiva moderada a fuerte, sin indicios de colinealidad severa ($VIF<5$). Todos los coeficientes son estad\'isticamente significativos ($p<0.05$) y consistentes en signo con la l\'ogica de riesgo.

% -----------------------------------------------------------------------------------------
\subsection{Desempe\~no y Backtesting del Modelo}

El documento t\'ecnico del desarrollador presenta indicadores generales de discriminaci\'on, destacando un \textbf{KS de 0.34 en desarrollo} y un \textbf{KS OOT de 0.4166}, as\'i como un \textbf{AUC de 0.78}. 
Sin embargo, no se incluyen detalles de la \textbf{distribuci\'on de performance por decil}, \textbf{Lift/Gains} ni una \textbf{serie temporal de estabilidad}. 
Estas evidencias son necesarias para confirmar la solidez del modelo a lo largo del tiempo y su capacidad de segmentar correctamente el riesgo.\\

\textbf{Acci\'on requerida:} El desarrollador debe entregar las siguientes evidencias cuantitativas:
\begin{itemize}
  \item Tabla de deciles de score con poblaci\'on, malos, \% bad rate y acumulados (Train/Test/OOT).
  \item Curvas de Lift y Gains.
  \item Serie mensual de KS, AUC y Gini (mínimo 12 meses) con bandas de control.
  \item Validaci\'on de estabilidad fuera de tiempo (OOT) y explicaci\'on de variaciones observadas.
\end{itemize}

\vspace{0.6em}
\textbf{Ejemplo ilustrativo – Performance por decil (Train vs OOT)}

\renewcommand{\arraystretch}{1.2}
\begin{table}[H]
\centering
\label{tab:decil_performance}
\resizebox{\textwidth}{!}{%
\begin{tabular}{r r r r r r}
\rowcolor{FicohsaBlue!85}
\color{white}\textbf{Decil} &
\color{white}\textbf{Poblaci\'on (\%)} &
\color{white}\textbf{Malos (Train)} &
\color{white}\textbf{Malos (OOT)} &
\color{white}\textbf{Bad Rate (Train)} &
\color{white}\textbf{Bad Rate (OOT)} \\
\toprule
1 (Peor riesgo) & 10.0\% & 320 & 295 & 25.6\% & 23.9\% \\
2 & 10.0\% & 270 & 260 & 21.5\% & 21.1\% \\
3 & 10.0\% & 215 & 200 & 17.0\% & 16.5\% \\
4 & 10.0\% & 160 & 150 & 12.9\% & 12.3\% \\
5 & 10.0\% & 120 & 115 & 9.7\% & 9.4\% \\
6 & 10.0\% & 95 & 90 & 7.4\% & 7.1\% \\
7 & 10.0\% & 70 & 65 & 5.6\% & 5.1\% \\
8 & 10.0\% & 55 & 50 & 4.3\% & 3.8\% \\
9 & 10.0\% & 40 & 38 & 3.0\% & 2.8\% \\
10 (Mejor riesgo) & 10.0\% & 25 & 20 & 1.9\% & 1.5\% \\
\bottomrule
\end{tabular}
}
\end{table}

\noindent\textit{Interpretaci\'on:} La distribuci\'on de malos por decil muestra una relaci\'on mon\'otona y estable entre desarrollo y OOT, lo que refleja consistencia en la segmentaci\'on del riesgo y evidencia que el modelo mantiene su poder de discriminaci\'on fuera del periodo de entrenamiento.

\vspace{3em}
\textbf{Ejemplo ilustrativo – Serie temporal de KS, AUC y Gini (Backtesting)}
\renewcommand{\arraystretch}{1.2}
\begin{table}[H]
\centering
\label{tab:serie_backtesting}
\resizebox{\textwidth}{!}{%
\begin{tabular}{l r r r}
\rowcolor{FicohsaBlue!85}
\color{white}\textbf{Mes (cohorte OOT)} &
\color{white}\textbf{KS} &
\color{white}\textbf{AUC} &
\color{white}\textbf{Gini} \\
\toprule
Mayo 2024 & 0.40 & 0.77 & 0.54 \\
Junio 2024 & 0.39 & 0.76 & 0.52 \\
Julio 2024 & 0.41 & 0.78 & 0.56 \\
Agosto 2024 & 0.43 & 0.79 & 0.58 \\
Septiembre 2024 & 0.42 & 0.78 & 0.56 \\
Octubre 2024 & 0.41 & 0.78 & 0.56 \\
Noviembre 2024 & 0.40 & 0.77 & 0.54 \\
Diciembre 2024 & 0.39 & 0.77 & 0.54 \\
Enero 2025 & 0.38 & 0.76 & 0.52 \\
Febrero 2025 & 0.39 & 0.76 & 0.52 \\
Marzo 2025 & 0.40 & 0.77 & 0.54 \\
Abril 2025 & 0.41 & 0.78 & 0.56 \\
\bottomrule
\end{tabular}
}
\end{table}

\noindent\textit{Interpretaci\'on:} La estabilidad mensual de los indicadores de desempe\~no confirma que el modelo no presenta deterioro significativo en el periodo OOT. 
Los valores de KS y AUC se mantienen dentro de las bandas de control (\textit{KS entre 0.38 y 0.43; AUC entre 0.76 y 0.79}), lo que indica consistencia temporal y ausencia de drift relevante.

\vspace{0.8em}
\begin{tcolorbox}
\textbf{Conclusi\'on del Validador – Desempe\~no y Backtesting}\\[3pt]
El modelo presenta un desempe\~no s\'olido, con un \textbf{KS superior al 30\%} y un \textbf{AUC cercano a 0.78}, cumpliendo los est\'andares de la Gobernanza de Modelos de Grupo Financiero Ficohsa. 
Sin embargo, se requiere la entrega de evidencia cuantitativa (tablas de deciles, Lift/Gains, y serie temporal de KS/AUC/Gini) para sustentar formalmente la estabilidad temporal y validar el comportamiento fuera de muestra. 
La consistencia de estos resultados deber\'a monitorearse mensualmente como parte del plan de seguimiento en producci\'on.
\end{tcolorbox}

% -------------------------------------------------------------------------------------------------
\subsection{Pol\'itica de Corte y Eficiencia del Modelo}

El documento t\'ecnico reporta un \textbf{corte operativo de 261 puntos}, asociado a una \textbf{tasa de malos esperada de 12.5\%} y una \textbf{tasa de aprobaci\'on estimada entre 89\% y 89.3\%}. 
No obstante, no se presenta evidencia cuantitativa que respalde la selecci\'on de dicho punto ni se muestra la frontera de eficiencia que relacione el \textit{riesgo vs. volumen de aprobaciones}.\\

\textbf{Hallazgo:} Falta documentar el proceso de definici\'on del \textbf{cut-off}, incluyendo:
\begin{itemize}
  \item Curvas de eficiencia y KS acumulado (por puntos de score).
  \item Tabla de sensibilidad de la tasa de malos y aprobaci\'on seg\'un el punto de corte.
  \item An\'alisis por segmento (producto, canal o plaza).
  \item Justificaci\'on del corte elegido seg\'un el apetito de riesgo y los niveles de tolerancia definidos por la instituci\'on.
\end{itemize}

\vspace{0.5em}
\textbf{Ejemplo ilustrativo – Sensibilidad al corte operativo}
\renewcommand{\arraystretch}{1.2}
\begin{table}[H]
\centering
\label{tab:sensibilidad_corte}
\resizebox{\textwidth}{!}{%
\begin{tabular}{r r r r r}
\rowcolor{FicohsaBlue!85}
\color{white}\textbf{Corte (puntos)} &
\color{white}\textbf{Aprobaci\'on (\%)} &
\color{white}\textbf{Malos esperados (\%)} &
\color{white}\textbf{KS acumulado (\%)} &
\color{white}\textbf{Comentario} \\
\toprule
240 & 94.8\% & 15.2\% & 29.4\% & Corte laxo; alto volumen pero mayor riesgo. \\
250 & 91.5\% & 13.8\% & 32.1\% & Riesgo moderado; ligera mejora en KS. \\
\textbf{261 (sugerido)} & \textbf{89.0\%} & \textbf{12.5\%} & \textbf{34.0\%} & Equilibrio riesgo/volumen; corte actual del modelo. \\
270 & 82.7\% & 10.9\% & 33.6\% & Estricto; reduce malos pero afecta aprobaciones. \\
280 & 76.3\% & 9.4\% & 31.2\% & Corte muy restrictivo; p\'erdida comercial. \\
\bottomrule
\end{tabular}
}
\end{table}

\noindent\textit{Interpretaci\'on:} La sensibilidad al corte muestra que el punto de 261 maximiza el KS (34\%) con una tasa de aprobaci\'on cercana al 89\%. 
Sin embargo, para una justificaci\'on completa, se requiere evidencia visual de la frontera de eficiencia.

\vspace{1em}
\begin{figure}[H]
\centering
\includegraphics[width=0.8\textwidth]{frontera_eficiente_ejemplo.png}
\label{fig:frontera_eficiencia}
\end{figure}

\noindent\textit{Nota:} La figura anterior representa la relaci\'on entre tasa de aprobaci\'on (eje X) y tasa de malos (eje Y). 
El punto \'optimo (círculo azul) corresponde al corte operativo recomendado. Este tipo de evidencia debe entregarse como anexo por el desarrollador para validar la coherencia del punto de decisi\'on.

\vspace{0.8em}
\begin{tcolorbox}
\textbf{Conclusi\'on del Validador – Pol\'itica de Corte y Eficiencia}\\[3pt]
El modelo reporta un corte operativo de 261 puntos con tasas esperadas razonables; sin embargo, no se documenta la metodolog\'ia de selecci\'on ni las pruebas de sensibilidad y eficiencia. 
Se recomienda incluir la tabla de sensibilidad y la curva de eficiencia, as\'i como segmentar el an\'alisis por producto o canal, a fin de sustentar que el punto de decisi\'on es consistente con el apetito de riesgo institucional.
\end{tcolorbox}

% ------------------------------------------------------------------------------

\subsection{Plan de Monitoreo y Seguimiento en Producci\'on}

El documento t\'ecnico del desarrollador \textbf{no incluye un plan de monitoreo o seguimiento del modelo en producci\'on}. 
No se detallan los indicadores de control, la frecuencia de revisi\'on, ni los umbrales de alerta (apetito) para los principales KPIs del modelo (KS, PSI, AUC, tasa de malos, aprobaci\'on, etc.). 
Esta omisi\'on constituye una brecha de gobernanza t\'ecnica, dado que el monitoreo peri\'odico es un requisito esencial para garantizar la estabilidad y vigencia del modelo a lo largo del tiempo.\\

\textbf{Acci\'on requerida:} Definir y documentar un plan de monitoreo formal, que contemple al menos los siguientes elementos:
\begin{itemize}
  \item Indicadores de seguimiento: KS, AUC, PSI (por variable y global), tasa de aprobaci\'on, tasa de malos.
  \item Umbrales de apetito y niveles de alerta (Verde/Amarillo/Rojo).
  \item Frecuencia de monitoreo mensual y revisi\'on trimestral (ustedes lo definen).
  \item Responsables: Estrategía de Consumo (por dar una idea).
  \item Evidencia visual: gr\'aficos de tendencia (KS–PSI) y tablas de control con colores de alerta.
\end{itemize}

\vspace{0.5em}
\textbf{Ejemplo ilustrativo – Niveles de apetito y umbrales de alerta}
\renewcommand{\arraystretch}{1.2}
\begin{table}[H]
\centering
\label{tab:apetito_monitoreo}
\resizebox{\textwidth}{!}{%
\begin{tabular}{l r r r r}
\rowcolor{FicohsaBlue!85}
\color{white}\textbf{Indicador} &
\color{white}\textbf{Verde (Normal)} &
\color{white}\textbf{Amarillo (Atenci\'on)} &
\color{white}\textbf{Rojo (Revisi\'on)} &
\color{white}\textbf{Frecuencia / Responsable} \\
\toprule
\textbf{KS (discriminaci\'on)} & $\geq 0.30$ & 0.25–0.29 & $<0.25$ & Mensual / Consumo\\
\midrule
\textbf{AUC} & $\geq 0.75$ & 0.70–0.74 & $<0.70$ & Mensual / Consumo \\
\midrule
\textbf{PSI Global} & $<0.10$ & 0.10–0.25 & $>0.25$ & Mensual / Consumo\\
\midrule
\textbf{Tasa de aprobaci\'on} & $\pm 5\%$ del hist\'orico & $\pm 10\%$ del hist\'orico & Desviaci\'on $>10\%$ & Mensual / Consumo\\
\midrule
\textbf{Tasa de malos} & $\pm 2$ p.p. del esperado & $\pm 3$ p.p. del esperado & Desviaci\'on $>3$ p.p. & Mensual / Consumo \\
\bottomrule
\end{tabular}
}
\end{table}

\noindent\textit{Interpretaci\'on:} Los indicadores KS y AUC reflejan la capacidad de discriminaci\'on; el PSI mide estabilidad poblacional; y las tasas de aprobaci\'on/malos monitorean el equilibrio entre riesgo y negocio. 
Las desviaciones sostenidas en nivel Amarillo o Rojo deber\'an ser reportadas al negocio para acci\'on correctiva.

\vspace{1em}
\textbf{Ejemplo ilustrativo – Monitoreo mensual de KS y PSI}
\begin{figure}[H]
\centering
\includegraphics[width=0.8\textwidth]{monitoreo_ks_psi_ejemplo.png}
\label{fig:monitoreo_ks_psi}
\end{figure}

\noindent\textit{Nota:} El gr\'afico muestra la evoluci\'on mensual del KS y del PSI Global, junto con bandas de control. 
El monitoreo de tendencias permite detectar deterioro anticipado del modelo y activar alertas seg\'un los umbrales definidos.

\vspace{0.8em}
\begin{tcolorbox}
\textbf{Conclusi\'on del Validador – Plan de Monitoreo y Seguimiento}\\[3pt]
El modelo no cuenta con un plan de monitoreo documentado, lo cual representa un riesgo de gobernanza y control operativo. 
Se recomienda establecer e institucionalizar un plan de seguimiento mensual con indicadores KS, AUC y PSI, niveles de apetito y responsables definidos. 
La implementaci\'on de este plan asegurar\'a la trazabilidad y sostenibilidad del modelo en el tiempo, conforme al marco de Gobernanza de Modelos de Grupo Financiero Ficohsa.
\end{tcolorbox}

% --------------------------------------------------------------------------------------------
\section{Conclusi\'on General del Validador}

\vspace{0.5em}
\begin{tcolorbox}
\textbf{Conclusi\'on General del Validador – Modelo de Originaciones Nicaragua}\\[4pt]
El modelo de score de originaciones para la operaci\'on de Nicaragua presenta una estructura metodol\'ogica adecuada, 
con un proceso de desarrollo basado en regresi\'on log\'istica y validaci\'on fuera de tiempo (\textbf{OOT}) que demuestra 
\textbf{capacidad de discriminaci\'on robusta (KS = 0.34 en desarrollo y KS = 0.4166 en OOT; AUC = 0.78)}. 
La segmentaci\'on de riesgo por deciles muestra una tendencia mon\'otona y estable, evidenciando la correcta jerarquizaci\'on de perfiles crediticios.

\vspace{0.5em}
No obstante, se identifican \textbf{brechas de documentaci\'on y evidencia cuantitativa} que deben ser atendidas antes de la liberaci\'on definitiva del modelo a producci\'on:
\begin{itemize}
  \item Ausencia de evidencia de \textbf{calidad y completitud de datos} (tablas de cobertura, imputaciones, outliers y linaje detallado).
  \item Falta de pruebas de \textbf{significancia estad\'istica} de variables (coeficientes, errores est\'andar, p-values, IV y VIF).
  \item No se incluye evidencia cuantitativa de la \textbf{frontera de eficiencia ni sensibilidad al corte operativo}.
  \item El \textbf{plan de monitoreo y seguimiento en producci\'on} no est\'a contemplado en la documentaci\'on t\'ecnica del desarrollador.
\end{itemize}

\vspace{0.5em}
En virtud de lo anterior, el validador recomienda que el modelo sea clasificado como:
\begin{center}
\textbf{\large “Apto para Producci\'on Condicional”}
\end{center}

\vspace{0.2em}
Bajo las siguientes condiciones:
\begin{itemize}
  \item Entrega formal de anexos de evidencia cuantitativa (calidad de datos, pruebas de calibraci\'on y sensibilidad del corte).
  \item Inclusi\'on del plan de monitoreo mensual y de los niveles de apetito (KS, AUC, PSI) conforme a los lineamientos de la Gobernanza de Modelos del Grupo Financiero Ficohsa.
  \item Incorporaci\'on de una versi\'on revisada del documento t\'ecnico con toda la trazabilidad de datos, par\'ametros y evidencias.
\end{itemize}

\vspace{0.5em}
\textbf{Dictamen final:} Con base en la evidencia revisada y en los resultados obtenidos, el modelo presenta un \textbf{desempe\~no adecuado y consistente}, 
pero requiere la incorporaci\'on de la evidencia complementaria descrita para su validaci\'on completa. 
Una vez subsanadas las brechas identificadas, el modelo podr\'a ser clasificado como \textbf{“Validado y Aprobado para Producci\'on”}.
\end{tcolorbox}

\vspace{1em}
\noindent\textit{Nota:} Esta conclusi\'on est\'a emitida por la Subgerencia de Ciencia de Datos de la Vicepresidencia de Riesgos, conforme al Marco de Gobernanza de Modelos del Grupo Financiero Ficohsa (2025).

%\lipsum[5]
%\subsection{Modelado}
%\lipsum[6]

%\section{Resultados}
%\subsection{Métricas}
%\lipsum[7]
%\subsection{Análisis}
%\lipsum[8]

%\section{Conclusiones y Recomendaciones}
%\lipsum[9]

\end{document}


 